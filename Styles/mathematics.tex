%----------------------------------------------------
%	MATHEMATICS
%----------------------------------------------------

% Tělesa, obory íntegrity a metrické prostory
\newcommand{\C}{\mathbb{C}}
\newcommand{\R}{\mathbb{R}}
\newcommand{\N}{\mathbb{N}}
\newcommand{\Q}{\mathbb{Q}}
\newcommand{\Z}{\mathbb{Z}}
\renewcommand{\L}[2]{L^{#1} \left( #2 \right)} % Lebesgueovy prostory

\newcommand{\vc}[1]{\boldsymbol{#1}} % vektor
\newcommand{\mat}[1]{\mathbf{#1}} % matice

\newcommand{\norm}[1]{\left \Vert #1 \right \Vert} % norma vektoru
\newcommand{\set}[1]{ \left \lbrace #1 \right \rbrace} % množina
\newcommand{\const}{\mathrm{konst}} % konstanta

\newcommand{\F}{\mathcal{F} } % Fourierova transformace
\newcommand{\La}{\mathcal{L}} % Laplaceova transformace

% Označení funkcí
\newcommand{\Res}[2]{\mathrm{Res}_{#1} \, #2 \,} % residuum
\newcommand{\sgn}{\, \mathrm{sign} \,} % signum
\newcommand{\tg}{\,\mathrm{tg}\,} % možné značení tangens


%Značení derivací a integrálů
\newcommand{\der}[2]{\frac{\mathrm{d}#1}{\mathrm{d}#2}} % obyčejná derivace
\newcommand{\pder}[2]{\frac{\partial #1}{\partial #2}} % parciální derivace
\newcommand{\ppder}[3]{\frac{\partial^2 #1}{\partial #2 \partial #3}} % parciální derivace

\newcommand{\tder}[3]{\left( \pder{#1}{#2} \right)_{#3 = \const}} % termodynamická derivace

\newcommand{\fder}[2]{\frac{\delta #1}{\delta #2}} % funkcionální derivace
\newcommand{\ffder}[3]{\frac{\delta^2 #1}{\delta #2 \delta #3}} % funkcionální derivace

\newcommand{\D}{\mathrm{d} } % integrační znamení
\newcommand{\DD}{\mathrm{D}} % absolutní derivace
\newcommand{\intR}{\int_{-\infty}^{\infty}} % integrál přes reálnou osu



% Značení posloupností, limit a sum
\newcommand{\sequence}[2]{ \left \lbrace #1 \right \rbrace_{#2=1}^\infty} % posloupnost
\newcommand{\sumnorm}[1]{\sum_{#1}^\infty} 
\newcommand{\limplus}[1]{\lim_{#1 \rightarrow + \infty}}
\newcommand{\limminus}[1]{\lim_{#1 \rightarrow - \infty}}


%Značení distribucí
\newcommand{\dual}[2]{\left \langle #1 ,#2 \right \rangle} % dualita
\newcommand{\tf}[1]{\mathcal{D}(#1)} % prostor testovacích funkcí
\newcommand{\dis}[1]{\mathcal{D'}(#1)} % prostor distribucí
\newcommand{\Schwartz}[1]{\mathcal{S}(#1)} % Schwartzův prostor
\newcommand{\weakstar}{\rightharpoonup^*} % slabá* konvergence 
\newcommand{\supp}{\mathrm{supp}\,} % nosič funkcí